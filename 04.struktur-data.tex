\chapter{STRUKTUR DATA}

Data-data yang digunakan untuk keperluan sistem aplikasi \textit{web service} PBB-P2 akan disimpan pada sistem basis data Oracle 11g di dalam basis data SISMIOP.

Tabel-tabel yang dibentuk dan digunakan pada sistem aplikasi ini adalah sebagai berikut :

\begin{enumerate}[1.]
  \item Tabel SPPT
  
  Tabel ini terdiri dari kolom berikut :
    
  \begin{longtable}{|r|l|c|r|}
    \hline
    \textbf{NO} & \textbf{KOLOM} & \textbf{TIPE} & \textbf{PANJANG} \\
    \hline \hline
    1 & KD\_PROPINSI & CHAR & 2 \\
    \hline
    2 & KD\_DATI2 & CHAR & 2 \\
    \hline
    3 & KD\_KECAMATAN & CHAR & 3 \\
    \hline
    4 & KD\_KELURAHAN & CHAR & 3 \\
    \hline
    5 & KD\_BLOK & CHAR & 3 \\
    \hline
    6 & NO\_URUT & CHAR & 4 \\
    \hline
    7 & KD\_JNS\_OP & CHAR & 1 \\
    \hline
    8 & THN\_PAJAK\_SPPT & CHAR & 4 \\
    \hline
    9 & SIKLUS\_SPPT & NUMBER & 2 \\
    \hline
    10 & KD\_KANWIL\_BANK & CHAR & 2 \\
    \hline
    11 & KD\_KPPBB\_BANK & CHAR & 2 \\
    \hline
    12 & KD\_BANK\_TUNGGAL & CHAR & 2 \\
    \hline
    13 & KD\_BANK\_PERSEPSI & CHAR & 2 \\
    \hline
    14 & KD\_TP & CHAR & 2 \\
    \hline
    15 & NM\_WP\_SPPT & VARCHAR2 & 30 \\
    \hline
    16 & JLN\_WP\_SPPT & VARCHAR2 & 30 \\
    \hline
    17 & BLOK\_KAV\_NO\_WP\_SPPT & VARCHAR2 & 15 \\
    \hline
    18 & RW\_WP\_SPPT & CHAR & 2 \\
    \hline
    19 & RT\_WP\_SPPT & CHAR & 3 \\
    \hline
    20 & KELURAHAN\_WP\_SPPT & VARCHAR2 & 30 \\
    \hline
    21 & KOTA\_WP\_SPPT & VARCHAR2 & 3O \\
    \hline
    22 & KD\_POS\_WP\_SPPT & VARCHAR2 & 5 \\
    \hline
    23 & NPWP\_SPPT & VARCHAR2 & 15 \\
    \hline
    24 & NO\_PERSIL\_SPPT & VARCHAR2 & 5 \\
    \hline
    25 & KD\_KLS\_TANAH & CHAR & 3 \\
    \hline
    26 & THN\_AWAL\_KLS\_TANAH & CHAR & 4 \\
    \hline
    27 & KD\_KLS\_BNG & CHAR & 3 \\
    \hline
    28 & THN\_AWAL\_KLS\_BNG & CHAR & 4 \\
    \hline
    29 & TGL\_JATUH\_TEMPO\_SPPT & DATE & \\
    \hline
    30 & LUAS\_BUMI\_SPPT & NUMBER & 12 \\
    \hline
    31 & LUAS\_BNG\_SPPT & NUMBER & 12 \\
    \hline
    32 & NJOP\_BUMI\_SPPT & NUMBER & 15 \\
    \hline
    33 & NJOP\_BNG\_SPPT & NUMBER & 15 \\
    \hline
    34 & NJOP\_SPPT & NUMBER & 15 \\
    \hline
    35 & NJOPTKP\_SPPT & NUMBER & 8 \\
    \hline
    36 & NJKP\_SPPT & NUMBER & 5,2 \\
    \hline
    37 & PBB\_TERHUTANG\_SPPT & NUMBER & 15 \\
    \hline
    38 & FAKTOR\_PENGURANG\_SPPT & NUMBER & 12 \\
    \hline
    39 & PBB\_YG\_HARUS\_DIBAYAR\_SPPT & NUMBER & 15 \\
    \hline
    40 & STATUS\_PEMBAYARAN\_SPPT & CHAR & 1 \\
    \hline
    41 & STATUS\_TAGIHAN\_SPPT & CHAR & 1 \\
    \hline
    42 & STATUS\_CETAK\_SPPT & CHAR & 1 \\
    \hline
    43 & TGL\_TERBIT\_SPPT & DATE & \\
    \hline
    44 & TGL\_CETAK\_SPPT & DATE & \\
    \hline
    45 & NIP\_PENCETAK\_SPPT & CHAR & 9 \\
    \hline
    \caption{Tabel SPPT}
  \end{longtable}
    
  Tabel ini digunakan untuk menyimpan informasi data-data tagihan untuk setiap objek pajak dan tahun pajak. Yang perlu diperhatikan adalah pada kolom STATUS\_PEMBAYARAN\_SPPT dimana bila isinya bernilai 0 (nol) maka tagihan tersebut belum terbayar, sementara bila bernilai 1 (satu) maka tagihan tersebut berarti sudah terbayarkan, namun bila isinya bernilai 2 (dua) maka akan menunjukkan bahwa objek pajak untuk tahun pajak tersebut telah dibatalkan tagihannya.
  
  \item Tabel PEMBAYARAN\_SPPT
  
  Informasi yang akan disimpan pada tabel PEMBAYARAN\_SPPT adalah sebagai berikut :
  
  \begin{longtable}{|r|l|c|r|}
    \hline
    \textbf{NO} & \textbf{KOLOM} & \textbf{TIPE} & \textbf{PANJANG} \\
    \hline \hline
    1 & KD\_PROPINSI & CHAR & 2 \\
    \hline
    2 & KD\_DATI2 & CHAR & 2 \\
    \hline
    3 & KD\_KECAMATAN & CHAR & 3 \\
    \hline
    4 & KD\_KELURAHAN & CHAR & 3 \\
    \hline
    5 & KD\_BLOK & CHAR & 3 \\
    \hline
    6 & NO\_URUT & CHAR & 4 \\ 
    \hline
    7 & KD\_JNS\_OP & CHAR & 1 \\
    \hline
    8 & THN\_PAJAK\_SPPT & CHAR & 4 \\
    \hline
    9 & PEMBAYARAN\_SPPT\_KE & NUMBER & 2 \\
    \hline
    10 & KD\_KANWIL\_BANK & CHAR & 2 \\
    \hline
    11 & KD\_KPPBB\_BANK & CHAR & 2 \\
    \hline
    12 & KD\_BANK\_TUNGGAL & CHAR & 2 \\
    \hline 
    13 & KD\_BANK\_PERSEPSI & CHAR & 2 \\
    \hline
    14 & KD\_TP & CHAR & 2 \\
    \hline
    15 & DENDA\_SPPT & NUMBER & 12 \\
    \hline
    16 & JML\_SPPT\_YG\_DIBAYAR & NUMBER & 15 \\
    \hline
    17 & TGL\_PEMBAYARAN\_SPPT & DATE & \\
    \hline
    18 & TGL\_REKAM\_BYR\_SPPT & DATE & \\
    \hline
    19 & NIP\_REKAM\_BYR\_SPPT & CHAR & 9 \\
    \hline
    \caption{Tabel PEMBAYARAN\_SPPT}
  \end{longtable}
  
  Tabel ini berfungsi untuk menyimpan informasi pembayaran yang terjadi berdasarkan tagihan yang muncul pada tabel SPPT.
  
  \item Tabel LOG\_TRX\_PEMBAYARAN
  
  Informasi yang disimpan pada tabel LOG\_TRX\_PEMBAYARAN adalah sebagai berikut :
  
  \begin{longtable}{|r|l|c|r|}
    \hline
    \textbf{NO} & \textbf{KOLOM} & \textbf{TIPE} & \textbf{PANJANG} \\
    \hline \hline
    1 & NOP & VARCHAR2 & 18 \\
    \hline
    2 & THN & VARCHAR2 & 4 \\
    \hline
    3 & NTPD & VARCHAR2 & 30 \\
    \hline
    4 & POKOK & NUMBER & \\
    \hline
    5 & NAMA\_WP & VARCHAR2 & 50 \\
    \hline
    6 & ALAMAT\_OP & VARCHAR2 & 150 \\
    \hline
    7 & MATA\_ANGGARAN & VARCHAR2 & 15 \\
    \hline
    8 & MA\_SANKSI & VARCHAR2 & 20 \\
    \hline
    9 & DENDA & NUMBER & \\
    \hline
    10 & PEMBAYARAN\_KE & NUMBER & 2 \\
    \hline
    11 & IP\_CLIENT & VARCHAR2 & 30 \\
    \hline
    \caption{Tabel LOG\_TRX\_PEMBAYARAN}
  \end{longtable}	
  
  Tabel ini digunakan untuk mencatat bahwa ada transaksi pembayaran yang telah selesai dan berhasil diproses oleh aplikasi.
  
  \item Tabel LOG\_REVERSAL
  
  Informasi yang disimpan pada tabel LOG\_REVERSAL adalah sebagai berikut :
  
  \begin{longtable}{|r|l|c|r|}
    \hline
    \textbf{NO} & \textbf{KOLOM} & \textbf{TIPE} & \textbf{PANJANG} \\
    \hline \hline
    1 & NOP & VARCHAR2 & 20 \\
    \hline
    2 & THN & VARCHAR2 & 4 \\
    \hline
    3 & NTPD & VARCHAR2 & 30 \\
    \hline
    4 & IP\_CLIENT & VARCHAR2 & 30 \\
    \hline
    \caption{Tabel LOG\_REVERSAL}
  \end{longtable}
  
  Tabel ini digunakan untuk mencatat aktivitas dari proses \textit{reversal} yang berhasil dilakukan.
\end{enumerate}