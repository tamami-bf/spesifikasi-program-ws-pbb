\chapter{FUNGSI YANG HARUS DILAKUKAN OLEH PROGRAM}

Fungsi-fungsi atau kebutuhan akan dibangunnya program atau sistem \textit{web service} ini adalah sebagai berikut :

\begin{enumerate}[1.]
  \item Mampu menampilkan informasi tagihan PBB-P2
  
  Dalam hal ini sesuai dengan parameter yang diberikan, yaitu berupa Nomor Objek Pajak (NOP) dan tahun pajak. Jadi begitu \textit{server} aplikasi menerima \textit{request} dari \textit{client}, tentunya \textit{client} mengirimkan NOP dan tahun pajak sebagai parameter masukkan bagi \textit{server}, kemudian atas dasar NOP dan tahun pajak tersebut, \textit{server} aplikasi akan melakukan komunikasi dengan \textit{server} basis data yang hasilnya berupa data tagihan yang akan dikirimkan ke \textit{client} dalam format JSON.
  
  \item Mampu mencatatkan transaksi pembayaran
  
  Untuk kondisi pencatatan pembayaran, maka \textit{client} akan disyaratkan mengirimkan Nomor Objek Pajak (NOP), Tahun Pajak, tanggal dilakukannya pembayaran, dan jam dilakukannya pembayaran. 
  
  Sistem aplikasi harus mampu melakukan perubahan pada kolom STATUS\_PEMBAYARAN\_SPPT pada tabel SPPT nilainya menjadi 1 (satu) yang artinya data tagihan untuk NOP dan tahun pajak tersebut telah terbayar.
  
  Sistem aplikasi juga harus mampu melakukan pengisian data pada tabel PEMBAYARAN\_SPPT. Termasuk mencatatkan aktivitasnya pada tabel LOG\_TRX\_PEMBAYARAN.
  
  \item Mampu melakukan pembatalan transaksi pembayaran
  
  Dalam beberapa kasus, kondisi pencatatan bisa saja mengalami kesalahan dan perlu koreksi terhadapnya, fungsi pembatalan transaksi pembayaran ditujukan untuk hal tersebut.
  
  Dalam hal ini, sistem aplikasi harus mampu melakukan \textit{reversal} atau mengembalikan data pencatatan pembayaran ke kondisi semula sebelum terjadinya pembayaran, artinya harus mampu mengubah isi dari kolom STATUS\_PEMBAYARAN\_SPPT di tabel SPPT menjadi 0 (nol) yang artinya tagihan belum terbayar, lalu menghapus isian pada tabel PEMBAYARAN\_SPPT, yang terakhir adalah mencatatkan aktivitasnya pada tabel LOG\_REVERSAL.
\end{enumerate}

Ketiga fungsi dasar itulah yang nantinya menjadi bahan dalam melakukan komunikasi dengan pihak Bank sebagai tempat pembayaran yang akan ditangani oleh sistem aplikasi ini.