\chapter{CAKUPAN DAN TUJUAN PROGRAM}

Sebagaimana judul dari buku ini, yaitu dokumentasi rincian sistem \textit{web services} sebagai pencatat transaksi pembayaran PBB-P2 (Pajak Bumi dan Bangunan Perdesaan dan Perkotaan), salah satu tujuan utamanya yaitu bagaimana agar setiap pembayaran yang diterima oleh Bank sebagai tempat pembayaran dapat tercatat secara langsung dan otomatis dalam Sistem Manajemen Informasi Objek Pajak.

Karena yang selama ini dilakukan adalah menggunakan 2 (dua) basis data yang terpisah secara fisik dan kegunaan sehingga timbulnya perbedaan data yang cukup banyak. Sebuah sistem basis data ditempat di Dinas Pendapatan dan Pengelolaan Keuangan Kabupaten Brebes sebagai tempat manajemen dan mengolah kondisi objek pajak dan besarnya piutang untuk tiap objek pajak, sementara sistem basis data yang lain berada pada Bank Pembangunan Daerah Jawa Tengah sebagai Bank tempat pembayaran yang mencatat objek-objek mana saja yang sudah terbayar dan objek-objek mana saja yang belum terbayar.

Kondisi perbedaan data muncul akibat berubahnya data pada sistem basis data pada DPPK Kabupaten Brebes apabila ada pelayanan PBB-P2 yang masuk dari masyarakat wajib pajak, karena keterbatasan sumber daya manusia yang akan menyita banyak waktunya apabila setiap ada perubahan pada basis data ini harus dipindahkan tagihannya ke basis data milik Bank BPD Jawa Tengah, maka diambil langkah pemindahan data akibat perubahan pada basis data di DPPK Kabupaten Brebes dilakukan di hari berikutnya.

Kondisi pencatatan pembayaran yang terjadi di Bank BPD Jawa Tengah pun demikian, data-data pembayaran untuk tiap objek baru dapat diunduh di hari berikutnya. Sehingga pada saat pencatatan pembayaran pada sistem basis data di DPPK Kabupaten Brebes, ada beberapa data yang berubah dan nilainya tidak sesuai dengan data pembayaran yang telah dilakukan di hari sebelumnya. Kondisi data yang telah tercatat terbayarkan pada sistem basis data Bank BPD Jawa Tengah pun tidak dapat diubah, sehingga perubahan yang terjadi pada basis data DPPK Kabupaten Brebes tidak dapat ditagihkan dan harus dikembalikan ke kondisi sebelum perubahan data terjadi.

Hal ini yang menjadi tujuan utama dibangunnya sistem pencatatan pembayaran otomatis agar kondisi-kondisi perbedaan data seperti diatas dapat diminimalkan.